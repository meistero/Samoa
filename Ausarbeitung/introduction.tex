
\section{Introduction}

Water waves are certainly among the most researched phenomena in natural sciences and engineering. Of particular interest to geoscientists and oceanographers is the accurate modeling of water wave propagation in tsunami scenarios. For this task, the non-linear shallow water equations have established themselves as an appropriate mathematical basis. They can be derived by depth-averaging the Navier-Stokes equations of fluid mechanics. By using kinematic boundary conditions for the free surface and the bottom and neglecting the vertical momentum equation, the vertical velocity component drops out of the shallow water equations. In addition, the assumption that the pressure is given only by the hydrostatic pressure (being the result of gravity) is typically used which yields the hydrostatic shallow water equations. 

In general, the assumption of a hydrostatic balance applies only to waves fulfilling the shallow water characteristic $h/\lambda <<1$, where $h$ is the height of the water column and $\lambda$ the wave length \cite{levin2008physics}. Typically, tsunami waves start as such shallow water waves but change to deep water waves when approaching the coast. While in the former phase, the influence of the vertical velocity component is negligible compared to the horizontal ones, its impact increases in the latter stage. Studies in \cite{horrillo} show that dispersive effects (dependency of the wave speeds on the frequency) have an impact on the run-up of tsunamis in coastal regions. Such effects are not part of the solutions produced by hydrostatic models \cite{horrillo}. 

Therefore, a different method based on modified non-hydrostatic shallow water equations has been successfully applied in various tsunami simulation frameworks, e.g. by \cite{fuchs},\cite{cui},\cite{walters} and \cite{stelling2003accurate}. In the non-hydrostatic shallow water equations, the vertical momentum equation is kept and the non-hydrostatic part of the pressure is now considered. At the same time, the depth-averaged nature of the hydrostatic shallow water equations is retained in order to preserve their computational efficiency. 

In \cite{samfass14extension}, this approach has been adapted to and integrated into the finite volume solver SWE for the shallow water equations on a uniform cartesian grid. The goal of this research is to similarly extend the simulation package \samoa (Space-filling curves
and Adaptive Meshes for Oceanic And Other Applications) which is, however, based on adaptive triangular meshes. In this work, we present the numerical model for solving the non-hydrostatic shallow water equations and its implementation in \samoa.

The remainder of this paper is organized as follows: first, a short review of the overall approach being used to solve the non-hydrostatic shallow water equations is given. Then, the discretization of the non-hydrostatic pressure equation in \samoa is explained in detail. Next, implementation-related aspects will be presented. Finally, the results obtained with the new model are shown and a short discussion of open questions for future research concludes this paper.

%introduce task, motivation for non-hydrostatic extension, refer to bachelor's thesis,short outline
%emphasis on project progression

